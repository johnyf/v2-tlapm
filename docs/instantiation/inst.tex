\documentclass[a4paper]{article}

\usepackage{fullpage}
\usepackage{amsmath}
\usepackage{amssymb}
\usepackage{listings}
\usepackage{parcolumns}
\usepackage{graphicx}
\usepackage{todonotes}

\title{An instantiation algorithm for TLA+ expressions}

\newcommand{\assignment}[1]{\{#1\}}
\newcommand{\inst}[2]{#1 {\leftarrow} #2}
\newcommand{\einst}[3]{#1 \stackrel{#2}{\Leftarrow} #3}
\newcommand{\tlaplus}[0]{{TLA+}}
\newcommand{\tla}[1]{#1}

\newcommand{\dor}{\textbf{or}}

\newcommand{\sidebyside}[2]{
    \begin{minipage}{0.45\linewidth}
      #1
    \end{minipage}
    \space{10pt}
    \begin{minipage}{0.45\linewidth}
      #2
    \end{minipage}
}

\lstdefinelanguage{tlaplus}{
  morekeywords= { MODULE, LET, IN, VARIABLE, VARIABLES, CONSTANT, CONSTANTS,
    ASSUME, NEW, PROVE, THEOREM, LEMMA, ENABLED, ==, INSTANCE, BY, DEF },
  sensitive=true,
  morecomment=[l]{\*},
  morecomment=[s]{(*}{*)},
  morestring=[b]",
  literate={~} {$\sim$}{1},
  columns=fullflexible,
  basicstyle=\small
}

\lstset{language=tlaplus}

\begin{document}

\maketitle

\section{Overview}
\label{sec:overview}
\tlaplus{} has two kinds of substitution: instantiation of modules, which
 preserves validity, and beta-reduction of lambda-expressions, which does not
 necessarily preserve validity. Following the notions of SANY, we distinguish
 three kinds of variables: temporal constants and temporal variables are
 bound by instantiation whereas formal parameters are bound by lambda
 abstraction. Furthermore, ENABLED binds all temporal variables within its
 argument which are in the context of the next state operator '.

 In the presence of unresolved instantiations or substitutions, it is often
 unclear, which operator binds the occurrence of a particular temporal variable:
 when we consider the expression
 \tla{$(\lambda\; fp : ENABLED\; (fp \# fp'))(x)$}, it seems that the variable
 $x$ is bound in the module. But in the beta-reduced form $ENABLED\; (x \# x')$
 of this expression, only the first occurrence of $x$ is bound by the
 module but its second (primed) occurrence is bound by ENABLED\footnote{To
 prevent complications from renaming, we will not assume alpha equivalence but
 will handle the renaming of bound variables explicitly.}.

 Furthermore, the two substitutions do not commute. For example, let us
 consider modules Foo and Bar:

\begin{parcolumns}{2}
\colchunk{
\begin{lstlisting}
---- MODULE Foo ----

VARIABLE x

E(u) == x' # u'
D(u) == ENABLED ( E(u) )

THEOREM T1 == D(x)

====
\end{lstlisting}
}
\colchunk{
\begin{lstlisting}
---- MODULE Bar ----

VARIABLE y

I == INSTANCE Foo with x <- y


THEOREM T2 == I!D(y)

====
\end{lstlisting}
}
\end{parcolumns}

\vspace{2mm}
\noindent
It looks like \tla{I!T1} and \tla{T2} talk about the same formula \tla{D(y)},
 but this is not the case: the first can be read as \tla{I!(D(y))}\footnote{This
 is not valid \tlaplus{} syntax.} and the second as \tla{(I!D)(y)}. In other
 words, it makes a difference if we beta-reduce first or if we instantiate
 first. Reducing D(x) first leads to \tla{ENABLED (x' \# x')}, which --
 following the renaming instuctions in ``Specifying Systems'' -- becomes
 \tla{ENABLED (\$x' \# \$x')} by the instantiation of \tla{x} with \tla{y},
 because primed occurrences of \tla{\$x} are bound by their enclosing ENABLED,
 not by the instantiation.

Instantiating first keeps the occurrence of the  formal parameter \tla{u} intact,
 leading to \tla{I!D(u) == ENABLED ( u \# \$x') } but reducing the application
 \tla{I!D(y)} leads to \tla{ENABLED (y' \# \$x')}. Now it is clear that
 \tla{I!(D(y))} is unsatisfiable while \tla{(I!D)(y)} is valid.\footnote{ Since
   one of the axioms  of \tlaplus{} is that (TRUE \# FALSE), we can always find
   a domain element different from $x$. In fact, the axioms of set-theory even
   enforce denumerable models.}.

In the following, we will develop algorithms for both kinds of substitutions.
 Since the occurrence of a lambda reducible expression (redex) as a subterm of
 an instantiation may block the evaluation of the latter, inner substitutions
 (i.e. those closer to the leaves of the term tree) must be evaluated before
 outer ones. Definitions will also need special consideration because
 we allow some of them to stay folded. But since a definition can contain
 substitutions, they can not be treated like temporal constants.\footnote{
   Actually, already a definition \tla{D(x) == e} contains a substitution
   because it is equivalent to D == LAMBDA x : e.
 }

\section{New Attempt: Explicit substitutions}

The original idea here is to represent both beta-reduction and instantiation
 explicitly in the term graph. Then the two formulas in the introduction
 could be written as D(u)\{u $\mapsto$ x\}[x $\mapsto$ y] and
 D(u)[x $\mapsto$ y]\{u $\mapsto$ y\}, where reduction is denoted by curly
 braces and instantiation is denoted by square braces. Actually, the SANY
 data-structures allow to write reduction as application to an abstraction:
 D(u)\{u $\mapsto$ x\} is then just \tla{(LAMBDA u: D(u))(x)}. Again, this
 is not legal \tlaplus{} but allowed by SANY.


\subsection{Datastructures}
\label{sec:ds}

\begin{tabular}{lp{0.4\textwidth}p{0.3\textwidth}}
node & content & comment \\
\hline
Module & list of constants, list of variables, & \\
       & list of instances, list of definitions & \\
(Temporal) Constant  & name, arity & Set of Constants CS \\
(Temporal) Variable  & name & arity == 0, Set of Variables VS \\
(Formal )Parameter & name, arity & Set of Paramters FP \\
Definition & name, arity, expression body & Set of Definitions DS,
                                            all FPs in body are bound \\
Expression  & constant & \\
          & \dor{} variable & \\
          & \dor{} parameter & \\
          & \dor{} definition & \\
          & \dor{} abstraction &\\
          & \dor{} application &\\
          & \dor{} substin & \\
%          & \dor{} selector &\\
Application & head expression, argument expression list & head.arity == list length\\
Abstraction & parameter, expression body & \\
SubstIn     & module, instantiation, expression body & the explicit instantiation node \\
Instantiation & list of assigments of variables/constants to expressions& \\
%Selector & instance, definition & the ! operator\\
\end{tabular}

The definition and instantiation elements do not contain arguments since they
 can always be rewritten in terms of abstractions: D(x) == F is equivalent to
 D == LAMBDA x : F and I(x)!D is equivalent to LAMBDA x : I!D. We write
 abstraction as $\lambda x : F$ and instantiation as $(\rho M\; with x_1
 \leftarrow t_1,\ldots,x_n \leftarrow t_n)!e$ where the variables / constants
 $x_i$ are exactly those declared in the module $M$.

To allow for more flexibility, we state the algorithm as a set of rewrite
 rules. The goal is to permute applications to abstractions further inside the
 term and resolve them at the leaves of the term tree (rules 1-3, 8 and 9) or,
 if the leaf is a folded definition, let the substitutions accumulate at the
 leaf (rules 4 and 10). Beta-reduction readily distributes over application
 (rule 5), but it needs special treatment when it is applied to an abstraction:
 in case there are no name clashes, $\lambda\; x$ can be evaluated before
 $\lambda\; y$ (rule 6), but if there are name clashes, we have to rename
 $y$ to a fresh variable $z$ first (rule 7).

\todo[inline]{Describe instantiations.}

We assume we have a set $Unfolded$ of definitions which are unfolded and
 a context tracing if we are inside ENABLED. When we denote meta-variables
 standing for any term as starting with ?, then the rewrite rules are:\\

\[
\begin{array}{ll@{\;\;\rightarrow\;\;}lll}
\\
  \multicolumn{3}{c}{$rewrite rule$}
  & \multicolumn{2}{c}{$side conditions$} \\
  \hline
  1&  (\lambda x : c)(?e) &  c & c \in CS \cup VS & \\
  2&  (\lambda x : x)(?e) &  ?e & x \in FP &\\
  3&  (\lambda x : y)(?e) &  y & ?y \in FP&\\
  4&  (\lambda x : D)(?e) &  (\lambda x : b)(?e) & D \in DS, b = ?D.body,
                                                   D \in Unfolded&\\
  5&  (\lambda x : ?f(?g))(?e) & ((\lambda x : ?f)(?e))(\lambda x : ?g)(?e)) &\\
  6&  (\lambda x : \lambda y : ?s)(?t)
   & (\lambda y : (\lambda x : ?s)(?t))
                          & y \not \in FV(?s) \cup FV(?t) & ***\\
  7&  (\lambda x : \lambda y : ?s)(?t)
   & (\lambda x : (\lambda z : (\lambda y : ?s)(z))(?t))
                          & y \in FV(s) \cup FV(t) , \\
  \multicolumn{3}{l}{} & z \not \in FV(?s) \cup FV(?t) & \\
  \hline
  8&  (\rho M\; with\; c \leftarrow ?s)!x & ?s & c \in CS \cup VS & * \\
  9&  (\rho M\; with\; c \leftarrow ?s)!x & x & x \in FP &\\
  10&  (\rho M\; with\; c \leftarrow ?s)!D & (\rho M\; with\; c \leftarrow s)!(b)
                          & D\in DS, D \in Unfolded, b = D.body&\\
  11 &  (\rho M\; with\; c \leftarrow ?s)!(?f(?g))
   & (\rho M\; with\; c \leftarrow ?s)!(?f)(\\
  \multicolumn{3}{r}{  (\rho M\; with\; c \leftarrow s)!(?g) )}
  & ?f \neq '\mbox{ or }?f\mbox{ outside of }EN  &\\
  12&  (\rho M\; with\; c \leftarrow ?s)!(?g')
   & ((\rho M\; with\; c \leftarrow \$c)!(?g) )'
                          & '\mbox{ inside of }EN, \$c \not \in CS\cup VS  &**\\
\end{array}
\]

Remarks:\\
\begin{tabular}{ll}
  * & all constants and variables of M are instantiated \\
  ** & this is unclean since it doesn't capture the difference between \\
     &  $EN (x\neq x' \land x=x')$ and $EN (x\neq x') \land EN (x=x')$ well\\
  \\
  & FP-substitution stops at folded definitions and CS/VS substitutions\\
  & CS/VS-substitution stops at folded definitions and FP-substitutions\\
\end{tabular}


\subsection{Termination}
\label{sec:termination}

We define a lexicographic order on the following measures:

\begin{enumerate}
\item Variable name clashes:
  \begin{itemize}
  \item $clash(c)=clash(v)=clash(fp)=0$
  \item $clash(s(t)) = clash(s) + clash(t)$ where $s$ is not abstraction
  \item $clash(\lambda x . s) t = clash(s)$ if $x \not \in FV(t)$
  \item $clash(\lambda x . s) t = clash(s) + 1$ if $x \in FV(t)$
  \item $clash(\lambda x . s) = clash(s)$ for the cases not covered above
  \end{itemize}

% \item Number of applications inside abstractions:
%   \begin{itemize}
%   \item $apps(c)=apps(v)=apps(fp)=0$
%   \item $apps(s(t)) = 1 + apps(s) + apps(t)$
%   \item $apps(\lambda x . t) = apps(t)$
% 
%   \item $absapps(c)=absapps(v)=absapps(fp)=0$
%   \item $absapps(s(t)) = absapps(s) + absapps(t)$
%   \item $absapps(\lambda x . t) = apps(t)+absapps(t)$
%  \end{itemize}

  
\item Deepest term under an abstraction which can be applied:
  \begin{itemize}
  \item $d(v)=d(c)=d(fp)=0$
  \item $d(\lambda x . s)=1 + d(s)$
  \item $d(s(t))= 1 + max(d(s), d(t))$
  \end{itemize}

  \begin{itemize}
  \item $da(v)=da(c)=da(fp)=0$
  \item $da(\lambda x . s)=1 + da(s)$
  \item $da(s(t))= \left\{
      \begin{array}{rl}
        1 + d(s) + max(da(s),da(t))&\mbox{if }s=\lambda x.r\\
        max(da(s),da(t))&\mbox{otherwise}\\
      \end{array}\right.$
  \end{itemize}

\end{enumerate}

\section{Open Problems}
\begin{itemize}
\item Confluence: I only checked overlaps of root position vs root position,
  non-root overlaps are possible
\item The ordering $da$ does not decrease in some cases (e.g.: wrap the redex of
  the *** rule into an application redex(e)). The reason is that outer pattern
  matches weigh heavier than inner ones:
  \begin{center}
    \includegraphics[width=4cm]{measure_ce1.pdf}
    \includegraphics[width=4cm]{measure_ce2.pdf}
  \end{center}
\end{itemize}

The weight of the abstraction on $x$ decreases as expected, but at the same time
 the weight of the abstraction on $y$ increases. Since $y$ is higher, its weight
 contributes more.

\section{Excursion: parametrized instantiations in SANY}
\label{sec:param-inst}

In SANY, parametrized instantuitions have a representation where the
 parameter is shifted over the instantiation. Let us consider the
 modules Foo and Bar:

\begin{parcolumns}{2}
\colchunk{
  \begin{lstlisting}
---- MODULE Foo ----

 VARIABLE a

 D(u) == u' # a'
 E(u) == D(u) \/ ENABLED D(u)

====
    \end{lstlisting}
}
\colchunk{
  \begin{lstlisting}
---- MODULE Bar ----
 EXTENDS Naturals

 VARIABLE x

 I(v) == INSTANCE Foo WITH a <- x+v

====
  \end{lstlisting}
}
\end{parcolumns}

The term \tla{I(x)!E(x)} is represented as \tla{I!E(x,x)} but they are
 not equivalent. In the meta-notation, they would be
 E\{u $\mapsto$ x\}[a $\mapsto$ x+v]\{v $\mapsto$ x\} vs. E\{u $\mapsto$ x,
 v $\mapsto$ x\}[a $\mapsto$ x+v] with the same consequences as in the
 introduction.

\vspace{2cm}
\todo[inline]{ compute the set of unfolded defs}
\todo[inline]{ $(ENABLED\; A) \Leftrightarrow  C$ used instantiated }
\end{document}
